\appendix
\chapter{Glossary}
\begin{itemize}
  \item \textbf{Black box machine}: Complex machine composed of a large number of physical parts carrying a certain number of internal functions and acting together in an organised fashion.
  \item \textbf{Description logics}: Family of formal languages used to represent the knowledge of a domain of interest.
  \item \textbf{Disease}: Impairement of a normal biological condition.
  \item \textbf{Drug}: Large or small molecule producing a pharmacological effect when administred in an organism. In this document, this word is interchangeably used with the word \emph{compound}.
  \item \textbf{Drug repositioning}: Identification of new therapeutic indications for known drugs. Also referred as \emph{drug repurposing}, \emph{re-profiling}, \emph{therapeutic switching} and \emph{drug re-tasking}.
  \item \textbf{OWL2 EL}: Profile of the Web Ontology Language implementing the axioms described in the description logic \dl{EL\textsuperscript{++}} family. 
  \item \textbf{Gene expression experiment}: Quantitative or qualitative characterisation of the number of messenger RNA molecules present in a biological system at a given time and location. This number reflects the expression or activity of genes of interest.
  \item \textbf{Indication}: Therapeutic usage of a drug, regulated by an authority.
  \item \textbf{Knowledge base}: Set of description logics axioms, entities and expressions.
  \item \textbf{Machine}: Assembly of parts functioning to meet a particular goal.
  \item \textbf{Mechanism of action}: Biochemical interaction that gives rise to the pharmacological effect of a drug.
  \item \textbf{Mode of action}: Broad activity of the molecule on an organism.
  \item \textbf{Ontology}: Specification of conceptualisation, formal representation of the knowledge of a domain of interest.
  \item \textbf{Off-label use}: Prescription of an approuved drug for a non-regulated indication.
  \item \textbf{Off-target}: Said to the secondary molecular entities affected by a drug. Off-targets are generally not desirable as they can produce side-effects.
  \item \textbf{Organism}: Assembly of molecules functioning as a stable whole.
  \item \textbf{Phenotype}: Set of characteristics or traits attributed to an organism.
  \item \textbf{Polypharmacology}: Potential of a drug to produce multiple pharmacological effects.
  \item \textbf{Reasoning}: Classification and consistency checking of a knowledge base. Also referred as \emph{classification}.
  \item \textbf{Semantic similarity}: Value reflecting the closeness between two concepts or entities.
  \item \textbf{Similar property principle}: Molecules with similar structures produce similar pharmacological effects.
\end{itemize}

\chapter{Abbreviations}
\begin{itemize}
  \item \textbf{CMap}: Connectivity Map.
  \item \textbf{DLs}: Description logics.
  \item \textbf{GO}: Gene Ontology.
  \item \textbf{GWAS}: Genome wide association study.
  \item \textbf{MoA}: Mode of action.
  \item \textbf{RDF}: Resource Description Framework.
  \item \textbf{SNP}: Single nucleotide polymorphism.
  \item \textbf{OBO}: Open biological and biomedical ontologies.
  \item \textbf{OWL}: Web Ontology Language.
\end{itemize}

\chapter{Peer-reviewed publications}
Chronological list of the peer-reviewed articles published in the course of the thesis. Authors are ranked as present on the original publications. Corresponding authors are underlined. Articles I directly published are openly accessible and a link to the PDF version is provided for the curious reader to follow. Articles resulting from collaborations and behind a paywall are identified as such.

\begin{itemize}
  \item \textbf{The CALBC RDF Triple store: retrieval over large literature content.} \\ Samuel Croset, Christoph Grabmueller, Chen Li, Silverstras Kavialauskas and \underline{Dietrich Rebholz-Schuhmann}. \\ \emph{Proceedings of the Workshop on Semantic Web Applications and Tools for Life Sciences 2010} (2010) \\ Document: \url{http://ceur-ws.org/Vol-698/paper6.pdf}
  \item \textbf{Exploring the Generation and Integration of Publishable Scientic Facts Using the Concept of Nano-publications.} \\ Amanda Clare, \underline{Samuel Croset}, Christoph Grabmueller, Senay Kafkas, Maria Liakata, Anika Oellrich and Dietrich Rebholz-Schuhmann \\ \emph{Proceedings of the 1st Workshop on Semantic Publishing 2011} (2011) \\ Document: \url{http://ceur-ws.org/Vol-721/paper-02.pdf}
  \item \textbf{OWL Representation of Drug Activities on Biological Systems.} \\ \underline{Samuel Croset} \\ \emph{Proceedings of the 3rd International Conference on Biomedical Ontology} (2012) \\ Document: \url{http://kr-med.org/icbofois2012/proceedings/ICBOFOIS2012ECS/SingleFiles/02_ICBO_FOIS_2012_ECS_Croset.pdf}
  \item \textbf{Argumentation to Represent and Reason over Biological Systems.} \\ \underline{Adam Wyner}, Luke Riley, Robert Hoehndorf, Samuel Croset \\ \emph{Lecture Notes in Computer Science - Information Technology in Bio-and Medical Informatics} (2012) \\ Document (not freely acessible): \url{http://link.springer.com/chapter/10.1007/978-3-642-32395-9_10}
  \item \textbf{Integration of the Anatomical Therapeutic Chemical Classification System and DrugBank using OWL and text-mining.} \\ \underline{Samuel Croset}, Robert Hoehndorf, Dietrich Rebholz-Schuhmann \\ \emph{Proceedings of the Ontologies In Biomedecine And Life Sciences 2012} (2012) \\ Document: \url{http://www.onto-med.de/obml/ws2012/obml2012report.pdf#page=23}
  \item \textbf{Brain: biomedical knowledge manipulation.} \\ \underline{Samuel Croset}, John P Overington, Dietrich Rebholz-Schuhmann \\ \emph{Bioinformatics} (2013) \\ Document: \url{http://bioinformatics.oxfordjournals.org/content/29/9/1238.full.pdf}
  \item \textbf{Brain, a library for the OWL2 EL profile.} \\ \underline{Samuel Croset}, John P Overington, Dietrich Rebholz-Schuhmann \\ \emph{Proceedings of the 10th OWL: Experiences and Directions Workshop} (2013) \\ Document: \url{http://ceur-ws.org/Vol-1080/owled2013_9.pdf}
  \item \textbf{A case study: semantic integration of gene–disease associations for type 2 diabetes mellitus from literature and biomedical data resources.} \\ \underline{Dietrich Rebholz-Schuhmann}, Christoph Grabmüller, Silvestras Kavaliauskas, Samuel Croset, Peter Woollard, Rolf Backofen, Wendy Filsell, Dominic Clark \\ \emph{Drug discovery today} (2013) \\ Document: \url{http://www.sciencedirect.com/science/article/pii/S1359644613003917}
  \item \textbf{The functional therapeutic chemical classification system.} \\ \underline{Samuel Croset}, John P Overington, Dietrich Rebholz-Schuhmann \\ \emph{Bioinformatics} (2013) \\ Document: \url{http://bioinformatics.oxfordjournals.org/content/early/2013/11/30/bioinformatics.btt628.full.pdf}
  \item \textbf{An ontology for drug-drug interactions.} \\ \underline{Maria Herrero-Zazo}, Janna Hastings, Isabel Segura-Bedmar, Samuel Croset, Paloma Martinez, Christoph Steinbeck \\ \emph{Proceedings of the Workshop on Semantic Web Applications and Tools for Life Sciences 2013} (2013) \\ Document: \url{http://ceur-ws.org/Vol-1114/Session3_Herrero-Zazo.pdf}
\end{itemize}

\chapter{Side-projects}
List of the finished side-projects done within the course of the thesis. More detailed descriptions of the projects are available online, at the provided URL.



- XMLBurger
- LifeReactor
- 

\begin{itemize}
  \item \textbf{PubMed Watcher}: \url{http://pubmed-watcher.org/}.
  \item \textbf{PubMed$^{2}$}: \url{http://pubmed-square.org/}.
  \item \textbf{How cool is my research?}: \url{http://howcoolismyresear.ch/}.
  \item \textbf{Redesign of the Apache Jena website}: \url{http://jena.apache.org/}.
  \item \textbf{ChEMBL Twitter Bot}: \url{https://twitter.com/ChEMBLBot}.
  \item \textbf{Life Reactor}: \url{https://twitter.com/ChEMBLBot}.
\end{itemize}
